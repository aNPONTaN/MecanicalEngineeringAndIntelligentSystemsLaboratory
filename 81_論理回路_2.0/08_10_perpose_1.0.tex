%
%
%
\section{目的・方法}
\subsection{目的}
\label{perpose}
我々は様々な現象を条件分岐で直し,定量化している.その値を計算機の処理できる値とし,計算機が人間を超える速度での演算を行っている.その処理についての基礎とそれを担う回路の使用法,設計法,動作について理解することが本実験の目的である..

%
%
%
\subsection{方法}
\label{method}
基本的な方法として,ブレッドボードに回路を実装し,通電後にどのような結果が出力されるかを調べた.

課題ごとの回路は必要な集積回路を用いて組み立てた.それぞれの素子は,次の集積回路を用いた.
\begin{itemize}
	\item NAND素子:74HC00
	\item AND素子:74HC08
	\item ExOR素子:74HC86
	\item J-Kフリップフロップ:74HC112
	\item Dフリップフロップ:74FC74
\end{itemize}

